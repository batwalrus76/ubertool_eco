\documentclass[10pt]{article}
\usepackage{graphicx}
\usepackage{endfloat}
\usepackage{amssymb}
\usepackage{amsmath}
\usepackage{setspace}
\usepackage[scientific-notation=true]{siunitx}
\usepackage[margin = 1in]{geometry}
\allowdisplaybreaks

\begin{document}
The KABAM model is used to estimate the potential bioaccumulation of hydrophobic organic pesticides in freshwater foodwebs and to estimate the risks to birds and mammals consuming the prey items from these freshwater food webs. The first part of the model calculate the parameters necessary to determine the pesticide concentration in tissues, bioaccumulation factors, bioconcentration factors, biomagnification factors and biota sediment accumulation factors for each trophic level in the aquatic food web. The KABAM model includes a trophic level representation for phytoplankton, zooplankton, benthic invertebrates, filter feeders, small fish, medium fish and large fish. The second part of the model estimates exposure and toxicological effects of a pesticide into risk estimates for mammals and birds consumging contaminated prey. Equations for the bioaccumulation of the food web are below:
\begin{align}
&    \hspace{-1.0in}  \text{Pesticide tissue residue for single trophic levels:} \nonumber \\
C_{B} & = \frac{k_{1} * (m_{o}*\phi*C_{wto} +m_{p}*C_{wdp})+k_{D}*\sum(P_{i}C_{Di})}{k_{2}+k_{E}+k_{G}+k_{M}} \nonumber \\
&    \hspace{-1.0in}  \text{parameters:} \nonumber \\
k_{1} & = \text{pesticide uptake rate constant through respiratory area (i.e. gills, skin} \nonumber \\
k_{2} & = \text{rate constant for elimination of the pesticide through the respiratory area (i.e. gills, skin} \nonumber \\
k_{D} & = \text{pesticide uptake rate constant for uptake through ingestion of food} \nonumber\\
k_{E} & = \text{rate constant for elimination of the pesticide through excretion of contaminated feces} \nonumber \\
k_{G} & = \text{organism growth rate constant} \nonumber \\
k_{M} & = \text{rate constant for pesticide metabolic transformation} \nonumber \\
m_{o} & = \text{fraction of respiratory ventilation involving overlying water} \nonumber \\
P_{i} & = \text{fraction of diet containing i (prey item)} \nonumber \\
\phi & = \text{fraction of the overlying water concentration of the pesticide that is freely dissolved} \nonumber \\
\vspace{0.5in}
&    \hspace{-1.0in}  \text{Calculation of available pesticide fraction in water:} \nonumber \\ 
\phi & = \frac{1}{1+(X_{POC}*\alpha_{POC}*K_{OW})+(X_{DOC}*\alpha_{DOC}*K_{OW})} \nonumber \\
&    \hspace{-1.0in}  \text{parameters:} \nonumber \\
X_{POC} & = \text{concentration of particulate organic carbon in water} \nonumber \\
X_{DOC} & = \text{concentration of dissolved organic carbon in water} \nonumber \\
\alpha_{POC} & = \text{proportionality constant to describe the similiarity of phase partitioning of POC}\nonumber\\ &\hspace{1.3em} \text{in relation to octanol}\nonumber\\
\alpha_{DOC} & = \text{proportionality constant to describe the similiarity of phase partitioning of DOC}\nonumber\\ &\hspace{1.3em} \text{in relation to octanol}\nonumber\\
&    \hspace{-1.0in}  \text{Calculation of pesticide concentration in the solid portion of the sediment:} \nonumber \\
C_{s} & = C_{SOC} * OC \nonumber \\
&    \hspace{-1.0in}  \text{where:} \nonumber \\
C_{SOC} & = C_{WDP} * K_{OC}\nonumber \\
&    \hspace{-1.0in}  \text{parameters:} \nonumber \\
C_{SOC} & = \text{normalized (for OC content) pesticide concentration in sediment} \nonumber \\
C_{WDP} & = \text{freely dissolved pesticide concentration in pore water} \nonumber \\
K_{OC} & = \text{organic carbon partition coefficient} \nonumber\\
OC & = \text{percent organic carbon in sediment} \nonumber\\
\end{align}
\end{document}