\documentclass[12pt, A4]{article}
\usepackage{graphicx}
\usepackage{endfloat}
\usepackage{amssymb}
\usepackage{amsmath}
\usepackage{setspace}
\usepackage[margin = 1in]{geometry}





\begin{document}

\title{Tier I Rice Model v1.0}

\author{Justley Harston}


\maketitle

%\doublespacing

The Formula of the Tier I Rice Model v1.0:

\begin{align}
C_w & = \frac{m_{ai}^\prime}{0.00105 + 0.00013K_d} \nonumber \\
\text{and, if appropriate:} \nonumber \\
K_d & = 0.01K_{oc} \nonumber \\
&    \hspace{-1.3in}  \text{where:} \nonumber \\
C_w & = \text{water concentration [$\mu$g/L]}  \nonumber \\
m_{ai}^\prime & = \text{mass of active ingredient applied per unit area [kg/ha]}  \nonumber \\
K_d & = \text{water-sediment partitioning coefficient [L/kg]} \nonumber \\
K_{oc} & = \text{organic carbon partitioning coefficient [L/kg]} \nonumber
\end{align} 

The Tier I Rice Conceptual Model:

\begin{align}
 \hspace{1.3in} C_w & = \frac{m_{ai}}{V_w + m_{sed}K_d}  \\
  &    \hspace{-1.3in}  \text{where:} \nonumber \\
C_w & = \text{water concentration [mass/volume]}  \nonumber \\
m_{ai} & = \text{mass of active ingredient applied to paddy [mass]}  \nonumber \\
V_w & = \text{volume of water column plus pore water [volume]}  \nonumber \\
m_{sed} & = \text{mass of sediment at equlibrium with water column [mass]} \nonumber \\
K_d & = \text{water-sediment partitioning coefficient [volume/mass]} \nonumber 
\end{align}

It is more customary to describe a rice paddy in terms of depth rather than volume or mass. Therefore, the following equations are defined:

\begin{align}
m_{sed} & = d_{sed} A \rho_b    \\ \nonumber \\
V_w & = d_w A + d_{sed} \theta_{sed}  A    \\ 
 &    \hspace{-1.3in}  \text{where:} \nonumber \\
d_{sed} & = \text{sediment depth [length]}  \nonumber \\
d_w & = \text{ water column depth [length]} \nonumber \\
A & = \text{area of the rice paddy [area]} \nonumber \\
\theta_{sed} & = \text{porosity of sediment [-]} \nonumber \\
\rho_b & = \text{bulk density of sediment [mass/volume] }\nonumber 
\end{align}

\begin{align}
m_{ai}^\prime  & =  \frac{m_{ai}}{A}    \\
 &    \hspace{-1.3in}  \text{where:} \nonumber \\
m_{ai}^\prime & = \text{mass applied per unit area [mass/area]} \nonumber
\end{align}

\begin{align}
 \hspace{1.3in} C_w =     \frac{m_{ai}^\prime}{d_w + d_{sed}  ( \theta_{sed}+ \rho_b K_d  )} 
\end{align}


JUST FOR LEARNING:  THIS IS T-REX
\begin{align}
C_t & = C_n e^{-kt} \\ \nonumber  \\
&    \hspace{-1.3in} \text{or in natural log form:} \nonumber \\ \nonumber \\
\text{ln}(C_t) & = \text{ln}(C_0) - kt
\end{align}











\end{document}